\documentclass{article}

\usepackage{microtype}
\usepackage{graphicx}
\usepackage{caption}
\usepackage{subcaption}
\usepackage{hyperref}
\usepackage{cite}

\usepackage{amsmath,amssymb,amstext}
\usepackage{amsfonts}
\usepackage{mathtools}

\newcommand{\state}{x}
\newcommand{\control}{v}
\newcommand{\setpoint}{y}
\newcommand{\error}{e}
\newcommand{\learningrate}{\alpha}
\newcommand{\td}[1]{\mathrm{d}#1}
\newcommand{\dt}{\mathrm{d}t}
\newcommand{\loss}{L}
\newcommand{\pdiv}[2]{\frac{\partial #1}{\partial #2}}
\newcommand{\tdiv}[2]{\frac{\td{#1}}{\td{#2}}}
\newcommand{\pihalf}{\frac{\pi}{2}}
\DeclareMathOperator{\const}{const}
\DeclareMathOperator{\sign}{sg}

\newcommand{\att}[2]{{#1}^{(#2)}}
\newcommand{\define}{:=}

\begin{document}
  \title{On the Problems of Using On-Line Gradient Descent to Optimize Adaptive PID Gains}
  \author{Erik Schultheis}
  \maketitle

  \clearpage
  \section{Introduction}
    In any kind of technical system $\state(t)$ that should automatically 
    be held in a state $\setpoint(t)$ (i.e. error $\error(t) := \setpoint(t) - \state(t)$ is 
    minimized) some form of control is necessary. 
    Except for the most simple cases, this requires a feedback loop which produces a 
    control output $\control(t) = \control(\state(t), \setpoint(t))$ that influences the system.


    One such algorithm that is often used due to its simplicity is the PID controller. 
    In this case $\control$ is a linear superposition of the current error, 
    the change rate in the error and the integral over all past errors
    \begin{align}
        \control(T) = P \error(T) + I \int_0^T \error(t) \dt + D \dot{\error}(T). 
        \label{eq:control_equation}
    \end{align}

    This algorithm works reasonably well for many systems.
    There are two drawbacks that one would like to improve: The algorithm requires 
    the three parameters $p, i, d$ to be tuned, and their optimal values may depend 
    on the current operating regime. Further, long time delays or strong nonlinearities in the 
    system reduce the effectiveness of PID control \cite{}.
    Therefore, extensions to PID control have been proposed that borrow ideas from neural networks 
    \cite{cong_novel_2005,yu_applying_2006,shu_pid_2000}.

    While there exist different variations for neural PIDs in the literature, 
    the basic idea remains the same: We consider the PID parameters $P, I, D$ as dynamical 
    parameters (weights in the language of neural networks) and try to optimize them such that a 
    loss function $L$ is minimized. 
    The loss function is chosen to be the squared error loss 
    \begin{align}
        L(T) = 0.5 \int_0^T \error(t)^2 \dt. \label{eq:l2loss}
    \end{align}
    In \cite{shu_pid_2000} the neurons additionaly possess a nonlinear activation function, in
     \cite{} the input to the neurons (the error term) is modelled by additional parameters 
    $\error = \alpha \setpoint + \beta \state$.


    In the following we want to show that this idea, though working well in certain circumstances, 
    has some fundamental flaws. For that, we will fist apply the algorithm to simple test cases to 
    demonstrate where the problems arise. Section \ref{sq:L2Loss} then follows with an analysis why 
    the squared error loss might be a bad metric to optimize. 






  \section{PIDNN control examples}
    For simple testing purposes, we  consider a linear second order plant of the form
    \begin{align}
        \ddot{\state} = a \dot{\state} + b \state,
    \end{align}
    which we want to control with an input $\control$ which acts as a force term
    \begin{align}
        \ddot{\state} = a \dot{\state} + b \state + c \control. \label{eq:qe_of_motion_system}
    \end{align}
    The control term is calculated as for a PID controller according to equation 
    \eqref{eq:control_equation}. 
    We call $w = (P, I, D)$ the parameter or weight vector. 
    We will be switching between discrete and continuous time formulations, 
    where $\att{t}{k} \define k \cdot \Delta t$ and in general $\att{x}{k} \define x(\att{t}{k})$.


    The update rule for the weights that is used in the literature \cite{cong_novel_2005} 
    (termed \emph{resilient backpropagation}) constitutes a very crude approximation to gradient 
    descend
    \begin{align}
        \att{w}{k+1} &= \att{w}{k} + \learningrate \left( \att{\error}{k} \cdot 
                        \sign \left(\frac{\att{\state}{k+1} - \att{\state}{k}}{\att{\control}{k} 
                        - \att{\control}{k-1}}\right) \cdot 
                        \pdiv{\att{\control}{k}}{w}\right), \label{eq:diffq_pidnn}
    \end{align}
    where $\learningrate$ denotes the learning rate. 

    The success of this method can be seen in figure \ref{fig:demo}, where we a classical PID 
    and its adaptive counterpart, for exactly the same configuration and initial weights. 
    The PIDNN managed to reduce the total loss from $5.15$ to $3.68$. However, the learning drives 
    the PIDNN towards a parameter regime where it becomes unstable, as can be seen in 
    \ref{fig:demo:diverging} where the setpoint following a rectangular function makes it impossible
    for the learning to stop. 

    \begin{figure}[tb]
    \begin{subfigure}[c]{0.5\textwidth}
        \includegraphics[width=1.0\textwidth]{../figures/demo1.pdf}
        \subcaption{Constant Setpoint}
        \label{fig:demo:converging}
    \end{subfigure}
    \begin{subfigure}[c]{0.5\textwidth}
        \includegraphics[width=1.0\textwidth]{../figures/demo2.pdf}
        \subcaption{Alternating Setpoint}
        \label{fig:demo:diverging}
    \end{subfigure}
    \caption{Application of the PIDNN algorithm to the system \eqref{eq:qe_of_motion_system}, for a 
    constant setpoint and a setpoint following a rect function. For reference the system development
    with a PID controller that has the same starting weights as the PIDNN is shown. While the PIDNN 
    works seemingly well for the constant target, it exhibits diverging behaviour in the second 
    case.}
    \label{fig:demo}
    \end{figure}














  \section{Stability Analysis}
    This example shows that the PIDNN can play havoc with system stability. The following sections 
    investigate why and under which conditions this happens. 

    The aim of finding optimal PID weights through gradient descent with respect to the loss 
    function eq. \eqref{eq:l2loss} can be cast in two different ways. Either aim at finding constant
    weights that minimize the loss over all time, or start to consider the weights as truly 
    dynamical parameters and use locally optimal values.



  \subsection{Global Minimization of \texorpdfstring{$L^2$}{L2} Loss}
    \label{sq:L2Loss}
    To investigate the objective that the loss function tries to optimize globally, we first derive the combined 
    equations of motion of \eqref{eq:qe_of_motion_system} and \eqref{eq:control_equation}.
    We will from now on assume a problem where the setpoint is fixed ($\setpoint(t) = \const$).
    This gives an inhomogeneous ODE with constant coefficients 
    \begin{align}
        \ddot{\state} &= a \dot{\state} + b \state + P (\setpoint - \state) + D \left( \dot{\setpoint} - \dot{\state} \right)\\ \nonumber
        &\eqqcolon 2\gamma  \dot{\state} - \beta \state + P \setpoint
    \end{align}
    We now want to find $P$ and $D$ such that the squared error between $\setpoint$ and $\state$ is minimal.
    Unfortunately, excluding the integrative term has the side effect of preventing the state to converge to the true target, so the error (integrated over all time) will become infinite and the error minimization will become meaningless. 
    One could in principle still perform the optimization by driving the fixed point to be as close as possible to the target value, which would be realised by $P \rightarrow \infty$ which is undesired. 
    
    Another approach is to postulate that the integrative term, if it were included in the model, would essentially converge to an almost constant value that is used exactly compensate the above mentioned deviation.
    Therefore, we will in the following set $b=0$ with the interpretation that this term has been canceled out by the integral contribution.

    The solution of the equation above can be formulated in terms of $\lambda=\sqrt{\gamma^2-P}$. This is either real
    (overdamped case) or imaginary (underdamped case), and we always assume that $\gamma<0$, i.e. that the system 
    dynamics do not blow up. We can then write the solution as
    \begin{align}
        \state(t) = A e^{(\gamma + \lambda) t} + B e^{(\gamma - \lambda) t} + \setpoint,   
    \end{align}
    where $A$ and $B$ are determined by $\state(0) = 0, \dot{\state(0)} = 0$ to
    \begin{align}
        A &= \frac{\gamma - \lambda}{2\lambda} \cdot \setpoint\\
        B &= -\frac{\gamma + \lambda}{2\lambda} \cdot \setpoint.
    \end{align}

    With that we can write down loss
    \begin{align}
        L &= \frac{1}{2} \int_0^{\infty} A^2 e^{2(\gamma + \lambda) t} + 2 A B e^{2\gamma t} + B^2 e^{2(\gamma - \lambda) t} \dt\\
          &= -\frac{A^2}{4(\gamma + \lambda)} - \frac{2 A B}{4\gamma} - \frac{B^2}{4(\gamma - \lambda)} \nonumber \\
          &= -\frac{A^2 (\gamma - \lambda) + B^2 (\gamma + \lambda)}{4(\gamma + \lambda)(\gamma - \lambda)} - \frac{A B}{2\gamma} \nonumber \\
          &= \left(\frac{(\gamma - \lambda) (\gamma + \lambda)}{2\gamma} - 
                   \frac{(\gamma - \lambda)^3 + (\gamma + \lambda)^3}{4(\gamma + \lambda)(\gamma - \lambda)}\right) 
             \frac{y^2}{4 \lambda^2}.
    \end{align}
    Merging the two fractions simplifies the expression to
    \begin{align}
        L &= \frac{y^2}{8 \gamma} \cdot \left( \frac{\lambda^2 - 5 \gamma^2}{\gamma^2 - \lambda^2}\right)\\
          &= \frac{y^2}{4 (D - a)} \cdot \left( \frac{(a - D)^2}{P} + 1\right) \label{eq:l2loss_simplified}
    \end{align}
    The value $\gamma$ does not depend on $P$, so the loss is 
    monotonically decreasing in $P$, meaning that the optimal value is $P \rightarrow \infty$.

    It is important to notice that this catastrophic behaviour is mostly a consequence of the 
    extreme simple model which allows the controlled system to react arbitrarily fast. Any 
    real physical system (or even simulation with discrete time step $\Delta t$) exhibits 
    delays and nonlinearities that prevent $P \rightarrow \infty$ from being the optimal
    solution. We can demonstrate that by a simple numerical simulation: Setting $a=y=1, D=2$
    in the above equations and simulating the corresponding system with a discrete time step
    of $dt=0.1$ for different values of $P$ we find a minimum in the loss for finite $P$, see 
    figure \ref{fig:l2_compare}.

    \begin{figure}[tb]
        \centering
        \includegraphics[width=0.75\textwidth]{../figures/compare_l2.pdf}
        \caption{Total $L^2$ loss (up to $T=1000$) for $a=y=1, 
        D=2$, calculated with formular \eqref{eq:l2loss_simplified} and simulated 
        with time discretizations $\dt=0.1$ and $\dt = 0.01$. We can see that the
        discretization introduces a minimum that moves towards infinity as $\dt$
        becomes smaller.}
        \label{fig:l2_compare}
    \end{figure}

    The above calculation is meant to show that one needs to be careful when optimizing 
    with respect to $L^2$ loss as this does not place any direct constraints on the 
    velocities and forces within the system, so lowering the loss does not necessarily
    guarantee a better performing system from an engineering standpoint.

    On more abstract grounds, we can argue that the $L^2$ loss is problematic because of its symmetries. Let $\error(t)$ be
    the error of a given system and $\loss[\error]$ be the $L^2$ loss functional. Defining 
    $\error^\prime(t) = \sqrt{\lambda} \error(t/\lambda)$ givens $\loss[\error] = \loss[\error^\prime]$, but from an 
    engineering point of view the system behaviour corresponding to $\error^\prime$ can be arbitrarily bad: The error
    decay is sped up by a factor of $\lambda$, but the overshoot is also increased, albeit only by the square root. 



  \subsection{Local Loss Minimization}
    We assume that the optimal weights change continuously, and over a time scale that is 
    larger than time step of the discretized system. Then we can use the following approximation: The optimal weights 
    $\att{w^*}{k}$ at timestep $k$ are those that minimize the loss over the time interval $[\att{t}{k}, \att{t}{k+1}]$. 
    These weights should then be also close to optimal for the next timestep so that 
    $\att{w}{k+1} \leftarrow \att{w^*}{k}$.

    We look at a discretized version of the plant. 
    We assume that the control input acts as a force term $g(\control, \state, \dot{\state})$, 
    which is assumed to be a nondecreasing function of the control input $\control$,
    in addition to the force produced 
    by the system $f(\state, \dot{\state}, t)$. This gives us 
    \begin{align}
        \att{\state}{k+1} \approx \att{\state}{k} + \att{\dot{\state}}{k} \Delta t + 0.5 \cdot \Delta t^2 \left( g(\att{\control}{k}, \att{\state}{k}, \att{\dot{\state}}{k}) + f(\att{\state}{k}, \att{\dot{\state}}{k}, t) \right)
    \end{align}
    in second order Taylor expansion. The loss over the interval $[\att{t}{k}, \att{t}{k+1}]$ is approximately
    $\att{L}{k} = 0.5 \left( {\att{\error}{k}}^2 + {\att{\error}{k+1}}^2 \right)$. 
    We want to find the optimal weights $\att{w}{k}$ at time $\att{t}{k}$ to get optimal control $\att{\control}{k}$. 
    Any other quantities at time $\att{t}{k}$ are considered constant.

    The chain rule thus gives
    \begin{align}
        \pdiv{\att{L}{k}}{\att{w_i}{k}} &= \att{\error}{k+1} \cdot \pdiv{\att{\error}{k+1}}{\att{\state}{k+1}} 
                                                             \cdot \pdiv{\att{\state}{k+1}}{\att{\control}{k}} 
                                                             \cdot \pdiv{\att{\control}{k}}{\att{w_i}{k}} \nonumber \\
         &= -\att{\error}{k+1} \cdot \pdiv{\att{\state}{k+1}}{\att{\control}k} 
                               \cdot \pdiv{\att{\control}{k}}{\att{w_i}{k}} \label{eq:discrete_gradient} \\
                        &= -\att{\error}{k+1} \cdot \frac{\Delta t^2}{2} 
                                              \cdot \pdiv{g(\att{\control}{k}, \att{\state}{k}, \att{\dot{\state}}{k})}{\att{\control}{k}} 
                                              \cdot \pdiv{\att{\control}{k}}{\att{w_i}{k}}.
    \end{align}
    We can rescale the loss by $0.5 \Delta t^2$, and the learning rate for gradient descent by the inverse. 
    For the control parameter $P$ this gives then in particular
    \begin{align}
        \pdiv{\att{L}{k}}{\att{P}{k}} = - \att{\error}{k} \cdot \att{\error}{k+1} \cdot g^\prime. \label{eq:diverging_p_local}
    \end{align}
    As $g^\prime \geq 0$ this value is negative except for cases when the sign of 
    $\att{\error}{k}$ and $\att{\error}{k+1}$ is different, 
    i.e. the state $\state$ crosses the setpoint $\setpoint$ in the considered interval. 
    Especially in the limit $\Delta t \rightarrow 0$, this will never happen and gradient descent will 
    always increase $P$.

    As was the case for global $L^2$ loss minimization, the algorithm again tries to act as fast as possible. 
    Since we are only working in a local approximation this may break global stability of the controlled
    system.

    The only way for the learning to stabilize is alternating $\att{\error}{k}$ which either means
    that $\state$ has to oscillate extremely strongly or that $\error$ is already quite low and 
    the control currently holds the system in the correct state. However, as soon as the setpoint 
    changes faster that the system can follow while still staying stable, $P$ will increase until
    either the system dynamics are fast enough to follow $\setpoint$ or the system loses its stability.



  \subsection{Approximation by Difference Quotient}
    The approach taken in the literature \cite{cong_novel_2005} is to use eq. \eqref{eq:discrete_gradient} but approximate 
    \begin{align}
        \pdiv{\att{\state}{k+1}}{\att{\control}{k}} \approx \frac{\att{\state}{k+1} - \att{\state}{k}}{\att{\control}{k} - \att{\control}{k-1}}.
    \end{align}
    Since this quantity easily diverges (the control remains constant, but the system continues to change due to inertia), 
    one often takes only the sign of the expression to get the direction of change. 

    The question remains whether this approximation eq. \eqref{eq:diffq_pidnn}, however badly motivated, manages to ensure
    the stability of the system.

    Going again to the limit $\Delta t \rightarrow 0$, we can write the weight update rule 
    \begin{align}
         \Delta w_i(t) = \error(t) \cdot \sign \left(\frac{\dot{\state}}{\dot{\control}} \right) \cdot \pdiv{\control}{w_i}.
         \label{eq:resilient_bp}
    \end{align}
    If we consider a rectangular target signal $\setpoint(t) = \mathrm{rect}(t/T)$, we have $\setpoint$ as a piecewise constant function. 
    To get an idea of the weight change that is induced by \eqref{eq:resilient_bp}, we assume that the state $\state$ oscillates with constant amplitude\footnote{For a well working controller this is a rather bad approximation. We will later lift that restriction, at the cost of not being able to solve the problem analytically.} and frequency around the setpoint, and integrate the change over a single period.
    \begin{align}
         \Delta w_i(t) &= \error(t) \cdot \sign \left( \frac{-\dot{\error}(t)}{p \dot{\error}(t) + d \ddot{\error}(t)} \right)  \cdot \pdiv{\control}{w_i} \label{eq:general_res_bp_wc}\\
         &= \sin(\omega t) \cdot  \mathrm{sg} \left(\frac{-\omega \cos{\omega t}}{p \cdot \omega \cos{\omega t} + d \cdot \omega^2 \sin{\omega t} } \right) \cdot \pdiv{\control}{w_i} \nonumber \\
         &=-\frac{ \sin(\omega t) }{\mathrm{sg}\left(p + d \omega \tan{\omega t}\right) } \cdot \pdiv{\control}{w_i}
    \end{align}
    We need to look at the sign in the denominator. This is negative iff
    \begin{align}
        \tan{\omega t} < -\frac{p}{d \omega}. \label{eq:sign_condition}
    \end{align}
    We get equality for $\omega t^* = \arctan \left(-\frac{p}{d \omega} \right)$. In the full interval $(-\pi/2, 3/2\pi]$ the denominator becomes negative for 
    \begin{align}
        \omega t \in (-\pi/2, \omega t^*) \cup (\pi/2, \pi/2+\omega t^*).
    \end{align}

    For the $p$ weight the partial derivative is $\partial_{p}\control(t) = \error(t) = \sin \omega t$. The total change during one oscillation is thus
    \begin{align}
        \int_{-\pi/2\omega}^{3\pi/2\omega} \hspace{-1em}\Delta p \, \dt &=  -\int_{-\pi/2}^{3\pi/2} \frac{ \sin^2(u) \td{u} }{\omega \sign \left(p + d \omega \tan{u}\right) }\\
    \end{align}
    By 
    \begin{align}
        \int_{a}^{b} \sin^2(t) \dt = \frac{1}{2} \left(b - a + \cos(a) \sin(a) - \cos(b) \sin(b) \right)
    \end{align}
    and $\sin^2(x + \pi) = \sin^2(x)$ we get
    \begin{align}
        \int_{-\pi/2\omega}^{3\pi/2\omega} \hspace{-1em}\Delta p \, \dt
        &= 2 \left(\int_{-\pi/2}^{u^*} \hspace{-1em} \sin^2(u) - \int_{u^*}^{\pi/2} \hspace{-1em} \sin^2(u) \right) / \omega \\
        &= \frac{1}{\omega} \left( u^* + \pihalf - \cos(u^*) \sin(u^*) - \left( \pihalf - u^* + \cos(u^*) \sin(u^*) \right)\right) \nonumber \\
        &= \frac{2}{\omega} \left(u^* - \cos(u^*) \sin(u^*) \right) \nonumber \\
        &= \frac{2 u^* - \sin(2 u^*)}{\omega}.
    \end{align}
    This is positive iff $u^*$ is positive. Setting $\kappa = \frac{p}{d \omega}$, we get $u^* < 0 \Leftrightarrow \kappa > 0$ by properties of the $\arctan$. As $p$, $d$ and $\omega$ are all positive, this always holds, so the weight change will always be negative.

    If we want to impose less stringent conditions on the form of the error, we can still get results numerically. Assume
    the error is modelled by a decaying oscillation
    \begin{align}
        \error(t) = A e^{-\gamma t} \cdot \cos(\omega t).
    \end{align}
    The prefactor $A$ can be taken out of the integral, and recaling of the time can eliminate the parameter $\omega$.
    For that set $\tilde{t} = \omega t$, $\tilde{\gamma} = \gamma / \omega$ so that
    \begin{align}
        \error(t) = A e^{\tilde{\gamma} \tilde{t}} \cdot \cos(\tilde{t}).
    \end{align}
    The total weight change \eqref{eq:general_res_bp_wc} can be written as
    \begin{align}
        \Delta w_i &= -\error(t) \sign \left( p + d \frac{\ddot{\error}(t)}{\dot{\error}(t)}\right)  \cdot \pdiv{\control}{w_i} \nonumber \\
                   &= -\error(t) \sign \left( 1 + \frac{d}{p}\cdot \frac{\ddot{\error}(t)}{\dot{\error}(t)}\right)  \cdot \pdiv{\control}{w_i}
                   \label{eq:weight_change}
    \end{align}
    This can be numerically evaluated for $p$ and $d$, which yields figure~\ref{fig:weight_change}. We can see that 
    whether the weights increase or decrease now depends on the value of $\gamma$, which makes it more difficult to 
    determine whether the system can be stable (since changes in $p$ and $d$ influence $\gamma$ and $r$).
    \begin{figure}[tb]
        \centering
        \includegraphics[width=0.75\textwidth]{../figures/wcc.pdf}
        \caption{Weight change according to eq. \eqref{eq:weight_change} for $p$ (solid) and $d$ (dashed), for different 
        values of $r = p/d$}
        \label{fig:weight_change}
    \end{figure}
    We can see that for slow decay $d$ will be increased and the decay will be slowed down further. For fast decay the 
    converse holds. Changes in $p$ only affect the amplitude of the oscillation but leave the decay unchanged.
    Therefore, it seems very unlikely that the weigh changes depicted in figure~\ref{fig:weight_change} will lead to 
    stable system behaviour in general.
    

  \subsection{Discussion}
    These results beg the question of how it is even possible to successfully apply a PIDNN as in figure \ref{fig:demo:converging}
    or \cite{cong_novel_2005,yu_applying_2006}. For that, not that $\partial_{\omega_i}\loss = \partial_\error L \cdot \partial_{\omega_i} \error$ and
    $\partial_\error \loss = \error$, which means that the weight change goes to zero if the error goes to zero.

    Thus, if the PIDNN manages to get the system into a state where it can follow the setpoint nearly perfecly, 
    learning will stop and the system will be stable. 
    It turns out that in this case, the adaptive algorithm in fact improves the transient response of the system. This
    is what is tested in e.g. \cite{yu_applying_2006}.

    However, if $\setpoint(t)$ is chosen in such a way that there exists no set of PID weights so that the system can 
    follow its setpoint, the weights will be continuously updated and by the arguments in the previous section the 
    system will loose its stability.

    This shows that the local optimization of the L2 loss (which in itself is already questionable, as per 
    section \ref{sq:L2Loss}) does not provide a way to get a stable system.

    %%%%%%%%%%%%%%%%%%%%%%%%%%%%%%%%%%%%%%%%%%%%%%%%%%%%%%%%%%%%%%%%%%%%%%%%%%%%
    %                       Lyapunov Argument
    %%%%%%%%%%%%%%%%%%%%%%%%%%%%%%%%%%%%%%%%%%%%%%%%%%%%%%%%%%%%%%%%%%%%%%%%%%%%
    In \cite{cong_novel_2005} an argument for the stability of PIDNN control is
    developed based on using $\att{J}{k}$ as a Lyapunov function and deriving 
    bounds for the learning rate which guarantee that 
    $\att{J}{k+1} < \att{J}{k}$.












  \section{Simulation Results}
    To complement the analytical findings of the previous section we performed a series of numerical experiments. 
    We started by looking deeper into the introductory example fig. \ref{fig:demo:diverging}. To get a more gradiual 
    change of weights we reduced to learning rate and compensated by an increased simulation time. Figure 
    \ref{subfig:sim:control} shows the control behaviour after 5000 time steps, shortly before the (Taylor) PIDNN becomes
    unstable. 

    The dynamics of the simulated system are very simple: 
    \begin{align}
        \ddot{\state} = -0.1 \state + \control. \label{eq:demo_system}
    \end{align}
    To make the results comparable with the analytical investigations above, the integral term has been fixed to $I=0.1$.
    The initial values where chosen to be $P=2.5$, $D=1.5$, the learning rate was set to $0.1$.

    In figure \ref{subfig:sim:weights} the corresponding weight development is plotted. It confirms the expected behaviour 
    of monotonically increasing $p$ for Taylor, but for Resilient the results differ significantly. While figure 
    \ref{fig:weight_change} predicts opposing signs for the changes in $p$ and $d$, \ref{subfig:sim:weights} both are 
    decreasing for most of the time. This discrepancy might be due to the fact that we assumed gradual weight change, 
    but $d$ changes quite rapidly at some points.
    \begin{figure}[htb]
    \begin{subfigure}[c]{0.5\textwidth}
        \includegraphics[width=1.0\textwidth]{../figures/compare_rect.pdf}
        \subcaption{Control with PID and both variations of PIDNN.}
        \label{subfig:sim:control}
    \end{subfigure}
    \begin{subfigure}[c]{0.5\textwidth}
        \includegraphics[width=1.0\textwidth]{../figures/compare_weights.pdf}
        \subcaption{Development of Weights. Dashed: Taylor, Solid: Resilient}
        \label{subfig:sim:weights}
    \end{subfigure}
    \caption{Simulation of PIDNN control of the system described by eq. \eqref{eq:demo_system}, 
             trying to follow a rectangular setpoint signal. For the system development only a short
              time window is shown, as indicated by the dotted lines.
             In this time frame the Taylor algorithm starts to diverge whereas Resilient still
             remains stable. However, at the end of the investigated time Resilient produces
             negative weights and it also diverges. For this simulation the $I$ weight has been held 
             constant.}
    \label{fig:weights}
    \end{figure}

    To get a better overview of the behaviour of the learning algorithm, figure \ref{fig:quiver} 
    depicts the weight change in dependence on the current weight.
    The graph shows that there is no possible way for the system to reach a stable fixed point or 
    limit cycle in the area where the error is small. 
    Doing a similar plot for the update rule based on the Taylor expansion
    does not reveal any new information, it just shows that $P$ and $D$ always increase.

    \begin{figure}[tb]
        \centering
        \includegraphics[width=1.0\textwidth]{../figures/quiver.pdf}
        \caption{Direction of weight change given current weights, for the system 
        \eqref{eq:demo_system} with Resilient PIDNN control and fixed $I$ weight. 
        The background color indicates the error that the given weights would generate were they 
        used with a static PID controller, where green indicates low error. 
        Data points which do not have an arrow indicate that the weight changes became too large 
        or even diverged.}
        \label{fig:quiver}
    \end{figure}

    \section{Discussion}
    In the preceding argumentation we can make out several shortcomings of the 
    PIDNN approach. At first, there is the problem that the $L^2$ loss does not 
    completely capture the notion of a successful system, as it does not 
    directly limit the velocities and forces that occur. However, we argued that
    this is mostly a theoretical problem as the system dynamics itself will 
    prevent arbitrarily fast movements. 

    The second problem arises from the fact that the system dynamics are unknown,
    but we would need them in order to correctly backpropagate the error to the 
    system control of the previous timestep. Both approaches taken, using a local
    Taylor approximation and replacing the actual derivative with a difference 
    quotient have proven to be not successful.

    Finally there is the so called "credit assigment problem". In a realistic 
    setting a control output $\att{\control}{k}$ might not affect the 
    system behaviour until many timesteps later, but our algorithm always 
    assumes an immediate effect. This is somewhat justified by the argument that
    both the system states $\att{\state}{k}$ and the control values 
    $\att{\control}{k}$ are temporally highly correlated sequences. 

    \section{TODO}
    \begin{itemize}
        \item A section about the Lyapunov stability argument found in some papers, and figure out 
                why it is wrong/incomplete.
                \cite{cong_novel_2005}
        \item Look at more complicated ideas to combine PIDs and NNs. 
                \cite{andrasik_-line_2004,zribi_new_2015}
        \item Numerical experiment in more complicated setting (maybe inverted pendulum as a 
              classical benchmark).
        \item Conclusion
    \end{itemize}


    \bibliographystyle{plain}
    \bibliography{lit.bib} 
\end{document}